\documentclass[15pt,a4paper]{report}
\usepackage[utf8]{vietnam}
\usepackage{amsmath}
\usepackage{amsfonts}
\usepackage{hyperref}
\usepackage{amsmath}
\usepackage{graphicx}
\usepackage{hyperref}
\usepackage{float} % Cần thiết cho tùy chọn [H]
\usepackage[utf8]{vietnam}
\usepackage{amsmath}
\usepackage{amsfonts}
\usepackage{amssymb}
\usepackage{graphicx}
\usepackage[left=1cm,right=1cm,top=1.5cm,bottom=1.5cm]{geometry}
\usepackage{blindtext}
\usepackage{titlesec}
\usepackage{graphicx}
\usepackage{multicol}
\usepackage{url}
\usepackage{matlab-prettifier}
\usepackage{caption}
\usepackage{arydshln}
\usepackage{mdframed}
\usepackage{amsthm}
\usepackage{listings}
\usepackage{xcolor}
\lstset{ 
	language=R,                % Ngôn ngữ
	basicstyle=\ttfamily\footnotesize, % Phông chữ
	%keywordstyle=\color{blue}, % Màu cho từ khóa
	commentstyle=\color{green!50!black}, % Màu cho chú thích
	stringstyle=\color{red},   % Màu cho chuỗi
	showstringspaces=false,    % Không hiển thị khoảng trắng trong chuỗi
	numbers=left,              % Hiển thị số dòng bên trái
	numberstyle=\tiny\color{gray}, % Phong chữ cho số dòng
	frame=single,              % Kẻ khung cho đoạn mã
	breaklines=true            % Tự động xuống dòng
}
\usepackage{graphicx}
\usepackage{pgfplots}
\fontsize{20pt}{15pt}
\usepgfplotslibrary{fillbetween}
\usetikzlibrary{patterns}


%%%%%%% code setup
\usepackage{listings}
\usepackage{caption}
\usepackage{xcolor}

% Không caption đánh số
\DeclareCaptionOption{lstlisting}{caption=false}{} 
\captionsetup[lstlisting]{labelformat=empty, position=bottom}

% (Tuỳ chọn) Màu nền nhẹ
\definecolor{mybg}{rgb}{0.94,0.94,0.94}

% Định nghĩa R
\lstdefinelanguage{R}{
	basicstyle=\ttfamily\small,
	keepspaces=true,
	showstringspaces=false,
	columns=fullflexible,
	upquote=true
}

% Định nghĩa Python
\lstdefinelanguage{Python}{
	basicstyle=\ttfamily\small,
	keepspaces=true,
	showstringspaces=false,
	columns=fullflexible,
	upquote=true
}

% Tắt hoàn toàn lề trái
\setlength{\parindent}{0pt}    % Không thụt lề đoạn văn
\setlength{\leftskip}{0pt}     % Không dịch trái toàn đoạn
\lstset{
	backgroundcolor=\color{mybg},
	breaklines=true,
	frame=none,
	numbers=none,
	xleftmargin=0pt,
	xrightmargin=0pt
}

%%%%%% end code setup

\begin{document}
\[
	\boxed{\huge \textbf{House Rental Price Analysis}}
\]
\section*{Data Preparation}
I have introduced the dataset \lstinline[language=R]|munichrent03| which is integrated in the R package \lstinline[language=R]|LinRegInteractive|, available at \href{https://github.com/taitran0102/House-rent-analysis/blob/main/README.md}{README.md} file. Therefore, I can simply load this dataset to begin the subsequent steps of the analysis.
\begin{lstlisting}[language=R]
library(LinRegInteractive)
data(munichrent03)
data <- munichrent03 
\end{lstlisting}
I began by examining the variable types to understand the structure of the dataset. 
\begin{lstlisting}[language=R]
> str(data)
'data.frame':	2053 obs. of  12 variables:
$ rent     : num  741 716 528 554 698 ...
$ rentsqm  : num  10.9 11.01 8.38 8.52 6.98 ...
$ area     : int  68 65 63 65 100 81 55 79 52 77 ...
$ rooms    : int  2 2 3 3 4 4 2 3 1 3 ...
$ yearc    : num  1918 1995 1918 1983 1995 ...
$ bathextra: Factor w/ 2 levels "no","yes": 1 1 1 2 2 1 2 1 1 1 ...
$ bathtile : Factor w/ 2 levels "yes","no": 1 1 1 1 1 1 1 1 1 1 ...
$ cheating : Factor w/ 2 levels "yes","no": 1 1 1 1 1 1 1 1 1 1 ...
$ district : Factor w/ 25 levels "All-Umenz","Alt-Le",..: 10 10 10 17 17 17 21 21 21 21 ...
$ location : Ord.factor w/ 3 levels "normal"<"good"<..: 2 2 2 1 2 1 1 1 1 1 ...
$ upkitchen: Factor w/ 2 levels "no","yes": 1 1 1 1 2 1 1 1 1 1 ...
$ wwater   : Factor w/ 2 levels "yes","no": 1 1 1 1 1 1 1 1 1 1 ...
> names(data)
'rent''rentsqm''area''rooms''yearc''bathextra''bathtile''cheating''district''location''upkitchen''wwater'
\end{lstlisting}
After that, I reviewed the distributions, value ranges, and identified any potential missing values.
\begin{lstlisting}[language=R]
> summary(data)
      rent            rentsqm            area           rooms      
Min.   :  77.31   Min.   : 1.470   Min.   : 17.0   Min.   :1.000  
1st Qu.: 389.95   1st Qu.: 6.800   1st Qu.: 53.0   1st Qu.:2.000  
Median : 534.30   Median : 8.470   Median : 67.0   Median :3.000  
Mean   : 570.09   Mean   : 8.394   Mean   : 69.6   Mean   :2.598  
3rd Qu.: 700.48   3rd Qu.:10.090   3rd Qu.: 83.0   3rd Qu.:3.000  
Max.   :1789.55   Max.   :20.090   Max.   :185.0   Max.   :6.000  

yearc      bathextra  bathtile   cheating        district      location   
Min.   :1918   no :1862   yes:1673   yes:1878   Neuh-Nymp: 177   normal:1205  
1st Qu.:1948   yes: 191   no : 380   no : 175   Lud-Isar : 161   good  : 803  
Median :1960                                    Au-Haid  : 139   top   :  45  
Mean   :1958                                    SchwWest : 137                
3rd Qu.:1973                                    Maxvor   : 132                
Max.   :2001                                    Laim     : 117                
(Other)  :1190                
upkitchen  wwater    
no :1903   yes:1981  
yes: 150   no :  72  

\end{lstlisting}
Note that the variables \lstinline[language=R]|rentsqm, rent| and \lstinline[language=R]|area| are related by the equation: \lstinline[language=R]|rent| = \lstinline[language=R]|rentsqm| $\times$ \lstinline[language=R]|area|. For example, at row 100, we have:

\begin{lstlisting}[language=R]
> round(data$rent[100]/data$area[100],2)  #compute rentsqm
11.3
> data$rentsqm[100]
11.3
\end{lstlisting}
Since \lstinline[language=R]|rentsqm| is a derived variable, I chose to exclude it and instead focus on total \lstinline[language=R]|rent|, which may better capture the underlying relationships with other features.
\begin{lstlisting}[language=R]
> data$rentsqm <- NULL
\end{lstlisting}
\section*{Exploratory Data Analysis}
\section*{Graphical Model Learning}
\section*{Inference and Querying}
\newpage

\[
\mathbb{P}(X<1,Y>1)=\int_{-\infty}^{1}\int_{1}^{+\infty}f(x,y) dx dy 
\]
\begin{figure}[H]
	\centering \includegraphics[width=0.6\textwidth]{unnamed-chunk-30-2.png}
\end{figure}
\end{document}